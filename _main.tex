\documentclass[]{book}
\usepackage{lmodern}
\usepackage{amssymb,amsmath}
\usepackage{ifxetex,ifluatex}
\usepackage{fixltx2e} % provides \textsubscript
\ifnum 0\ifxetex 1\fi\ifluatex 1\fi=0 % if pdftex
  \usepackage[T1]{fontenc}
  \usepackage[utf8]{inputenc}
\else % if luatex or xelatex
  \ifxetex
    \usepackage{mathspec}
  \else
    \usepackage{fontspec}
  \fi
  \defaultfontfeatures{Ligatures=TeX,Scale=MatchLowercase}
\fi
% use upquote if available, for straight quotes in verbatim environments
\IfFileExists{upquote.sty}{\usepackage{upquote}}{}
% use microtype if available
\IfFileExists{microtype.sty}{%
\usepackage{microtype}
\UseMicrotypeSet[protrusion]{basicmath} % disable protrusion for tt fonts
}{}
\usepackage[margin=1in]{geometry}
\usepackage{hyperref}
\hypersetup{unicode=true,
            pdftitle={Real Estate Market Analysis},
            pdfauthor={Maciej Beręsewicz},
            pdfborder={0 0 0},
            breaklinks=true}
\urlstyle{same}  % don't use monospace font for urls
\usepackage{natbib}
\bibliographystyle{apalike}
\usepackage{longtable,booktabs}
\usepackage{graphicx,grffile}
\makeatletter
\def\maxwidth{\ifdim\Gin@nat@width>\linewidth\linewidth\else\Gin@nat@width\fi}
\def\maxheight{\ifdim\Gin@nat@height>\textheight\textheight\else\Gin@nat@height\fi}
\makeatother
% Scale images if necessary, so that they will not overflow the page
% margins by default, and it is still possible to overwrite the defaults
% using explicit options in \includegraphics[width, height, ...]{}
\setkeys{Gin}{width=\maxwidth,height=\maxheight,keepaspectratio}
\IfFileExists{parskip.sty}{%
\usepackage{parskip}
}{% else
\setlength{\parindent}{0pt}
\setlength{\parskip}{6pt plus 2pt minus 1pt}
}
\setlength{\emergencystretch}{3em}  % prevent overfull lines
\providecommand{\tightlist}{%
  \setlength{\itemsep}{0pt}\setlength{\parskip}{0pt}}
\setcounter{secnumdepth}{5}
% Redefines (sub)paragraphs to behave more like sections
\ifx\paragraph\undefined\else
\let\oldparagraph\paragraph
\renewcommand{\paragraph}[1]{\oldparagraph{#1}\mbox{}}
\fi
\ifx\subparagraph\undefined\else
\let\oldsubparagraph\subparagraph
\renewcommand{\subparagraph}[1]{\oldsubparagraph{#1}\mbox{}}
\fi

%%% Use protect on footnotes to avoid problems with footnotes in titles
\let\rmarkdownfootnote\footnote%
\def\footnote{\protect\rmarkdownfootnote}

%%% Change title format to be more compact
\usepackage{titling}

% Create subtitle command for use in maketitle
\newcommand{\subtitle}[1]{
  \posttitle{
    \begin{center}\large#1\end{center}
    }
}

\setlength{\droptitle}{-2em}
  \title{Real Estate Market Analysis}
  \pretitle{\vspace{\droptitle}\centering\huge}
  \posttitle{\par}
  \author{Maciej Beręsewicz}
  \preauthor{\centering\large\emph}
  \postauthor{\par}
  \predate{\centering\large\emph}
  \postdate{\par}
  \date{2017-05-09}

\usepackage{booktabs}
\usepackage{amsthm}
\makeatletter
\def\thm@space@setup{%
  \thm@preskip=8pt plus 2pt minus 4pt
  \thm@postskip=\thm@preskip
}
\makeatother

\begin{document}
\maketitle

{
\setcounter{tocdepth}{1}
\tableofcontents
}
\chapter{Prerequisites}\label{prerequisites}

\chapter{Introduction}\label{intro}

\chapter{Data sources on real estate market}\label{data-sources}

\chapter{Internet data sources}\label{internet-data-sources}

\chapter{Characteristics of Real Estate Market in Poland and
Europe}\label{characteristics-of-real-estate-market-in-poland-and-europe}

\chapter{Descriptives statistics as basic method of analysis of Real
Estate
Market}\label{descriptives-statistics-as-basic-method-of-analysis-of-real-estate-market}

\chapter{Data cleaning techniques and outliers
detection}\label{data-cleaning-techniques-and-outliers-detection}

\section{Outliers and influence
measures}\label{outliers-and-influence-measures}

\textbf{Leverage} given by

\[
h_i = \frac{1}{n} + \frac{(X_i - \overline{X})}{\sum(X_i - \overline{X})^2}
\]

\textbf{Studentized residuals} given by

\[
e_{i}^{*}=\frac{e_i}{S_{e(-1)}\sqrt{1-h_i}}
\]

where: \(e_i\) - residual, \(S_{e(-1)}\) - standard error of the
regression without i-th observation. Studentized residuals follow
t-distribution with \(n-k-2\) degress of freedom.

\textbf{Cook distance} \[ 
D_i=\frac{e_i}{k+1}\frac{h_i}{1-h_i}
\]

where \(k\) -- number of dependent variables,
\(h_i = \frac{1}{n} + \frac{(X_i - \overline{X})}{\sum(X_i - \overline{X})^2}\)
, \(MSE=\frac{1}{n}\sum_{i=1}^n(\hat{Y}_i-Y_i)^2\) outliers meet:

\[
D_i>\frac{4}{n-k-1}
\]

\textbf{DFBETA} measures change in estimates of regression parameters
when we remove one observation

\[
DFBETA_i=(\sum_{i \in s} \mathbf{x}_i\mathbf{x}_i^T)^{-1}\mathbf{x}_i\frac{e_i}{1-\mathbf{x}_i^T(\sum_{i \in s} \mathbf{x}_i\mathbf{x}_i^T)^{-1}\mathbf{x}_i}
\]

\textbf{DFBETAS} - standarised version of DFBETA. Measures influence in
units of standard error of regression.

\[ 
DFBETAS_i=\frac{\hat{\mathbf{\beta}}-\hat{\mathbf{\beta}}_{(-i)}}{\sqrt{MSE_{(-i)}}}=\frac{DFBETA_i}{\sqrt{MSE_{(-i)}}}
\]

Outliers meet:

\begin{itemize}
\tightlist
\item
  \(|DFBETAS_i|>2\) - small samples
\item
  \(DFBETAS_i>\frac{2}{\sqrt{n}}\)
\end{itemize}

\textbf{DFFITS} -- measures global difference between model with and
without \emph{i} observation.

\[
DFFITS_i=\frac{e_i\sqrt{\frac{h_i}{1-h_i}}}{\sqrt{MSE_{(-i)}}\sqrt{{1-h_i}}}
\]

Outliers meet \(|DFFITS_i| > 2\sqrt{\frac{p+1}{n-k-1}}\)

\textbf{CovRatio} -- measures influence on variance of regression
coefficients

\[
COVRATIO_i=\frac{1}{(\frac{n-k-2+t_i^2}{n-k-1})^{k+2}}\frac{1}{(1-h_i)}
\]

where \(h_i\) is the same as in Cook's distance, \(t_i\) is defined

\[
t_i=\frac{e_i}{\sqrt{MSE_{(-i)}}\sqrt{{1-h_i}}}
\]

Interpretation:

\begin{itemize}
\tightlist
\item
  \(COVRATIO_i < 1\) - elimination of \(i\) th unit/observation will
  reduce standard errors of regression coefficients
\item
  \(COVRATIO_i > 1\) - elimination of \(i\) th unit/observation will
  increase standard errors of regression coefficients
\end{itemize}

it is suggested to use sample size dependent thresholds

\[
|COVRATIO_i-1| > 3(k+1)/n
\]

\chapter{Indices}\label{indices}

\chapter{Regression}\label{regression}

\chapter{GIS Tools}\label{gis-tools}

\chapter{Spatial models}\label{spatial-models}

\chapter{Forecasting}\label{forecasting}

\chapter{Placeholder}\label{placeholder}

\chapter{Additional topics}\label{additional-topics}

\bibliography{packages.bib,book.bib}


\end{document}
